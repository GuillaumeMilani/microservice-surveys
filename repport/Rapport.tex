\newcommand\auteur{Milani Guillaume, Clavien Tony, Luthier Gabriel \& Guillod Maxime}
\newcommand\cours{AMT}
\newcommand\ecole{IL - TIC - HEIG-VD}
\newcommand\domaine{Microservice}
\newcommand\titre{Surveys}
\newcommand{\inlineCode}[1]{\lstinline[basicstyle=\ttfamily\color{black}]|#1|}

\newcommand{\f}[1]{\lstinputlisting[caption=#1]{../#1}}


\documentclass[a4paper,11pt]{article}
%
\author{\auteur}
\title{\titre}
\date{\today}

\usepackage{fancyhdr}
\usepackage{graphicx}
\usepackage{amsmath}
\usepackage{listings}
\usepackage{multicol}
\usepackage{color}
\usepackage{enumerate}
\usepackage[utf8]{inputenc}
\usepackage[frenchb]{babel}
\usepackage{float}
\usepackage{listingsutf8}
\usepackage{geometry}
\usepackage{amssymb,mathtools,pifont} 
\usepackage{enumitem}
\usepackage{xspace}
\geometry{verbose,tmargin=2.5cm,bmargin=2cm,lmargin=2cm,rmargin=2.5cm}
\selectlanguage{frenchb}
\frenchbsetup{StandardLists=true}
\DeclareGraphicsExtensions{.pdf,.png,.jpg}
\setlength\parindent{0pt}

\usepackage{algorithm}
\usepackage[noend]{algpseudocode}
%
\floatname{algorithm}{Algorithme}
\algrenewcommand\algorithmicrequire{\textbf{Données :}}
\algrenewcommand\algorithmicensure{\textbf{Résultat :}}
%
\algnewcommand\algorithmicto{\textbf{to}}
\algrenewtext{For}[3]{\algorithmicfor\ #1 $\gets$ #2 \algorithmicto\ #3 \algorithmicdo}
%
\algrenewcommand{\algorithmiccomment}[1]{\hfill{\color{blue}$\triangleright$ #1}}

% headers & footers
\pagestyle{fancy}

\lhead{\domaine}
\rhead{\titre}

\lfoot{\auteur}
\cfoot{}
\rfoot{\thepage}
\lstset{
	inputencoding=utf8/latin1,
	breaklines=true, 
	language=bash,  
	basicstyle=\small, 
	numbers=left,   
	firstnumber=1,
	numberfirstline=true, 
	keywordstyle=\color{blue}\ttfamily,
	stringstyle=\color{red}\ttfamily,
	commentstyle=\color{black}\ttfamily,
	morecomment=[l][\color{magenta}]{\#},
	showstringspaces=false,
	xleftmargin=1cm,
	tabsize=4,
}

\begin{document}
	
	% Entete première page
	\thispagestyle{empty}
	%
	\noindent \cours \hfill \ecole{} \newline
	\noindent \auteur \hfill \today \newline
	\hrule
	\vspace{7mm}
	\noindent {\large \bf \domaine } \hfill \titre {\large \bf }\\[3mm]
	\hrule
	
	\section{Introduction}
	Vous trouverez dans ce document toutes les informations relatives à l'installation ainsi qu'au lancement de notre micro-service \textit{Surveys}. 
	
	\section{Pré-requis}
		\subsection{Docker}
		Pour commencer, vous devez installer sur votre machine \textit{Docker} afin de pouvoir lancer en toute simplicité notre service \textit{Surveys}. En effet, nous utilisons \textit{Docker} afin de lancer notre base de données \textbf{MongoDB}, éventuellement \textit{Mongo-Express} afin de visualiser le contenu de notre base de données, et bien entendu, \textit{Surveys}, notre micro-service.
		
		\subsection{Maven}
		Si vous souhaitez installer notre service en compilant le code source, vous aurez besoin d'avoir dockerBuild.sh\textit{Maven} installé sur votre machine pour la compilation de notre \textit{.jar}.
		
	\section{Installation code source}
		Pour créer un exécutable \textit{.jar} afin de pouvoir lancer notre service, vous pouvez lancer notre script \textit{bash} afin de compiler automatiquement ce dernier.
		
		\f{dockerBuild.sh}
	
	\section{Lancement du service}
	\subsection{Script}
	La méthode la plus simple pour lancer notre service \textit{Surveys} avec la base de donnée \textit{Mongo} ainsi que \textit{Mongo-Express} pour la visualisation des données, et de lancer le script \textbf{bash} suivant : 
	
	\f{dockerRun.sh}
	
	\subsection{Docker-compose}
	Vous pouvez également utiliser notre fichier \textit{docker-compose.yml} afin de lancer les services que vous souhaitez.
	
	\f{docker-compose.yml}
	
	En utilisant la commande suivante, vous pouvez par exemple lancer uniquement \textit{Mongo} ainsi que notre service \textit{Surveys} : 
	\inlineCode{docker-compose mongo server}


\end{document}

